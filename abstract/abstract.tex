\documentclass{scdpg}
\begin{document}
\scBookLanguage{de}
\begin{scAbstract}
% \scNoUseTeX
\scLanguage{de}
\scTitle{Multi-repräsentationale Lernaufgaben zur Vektoranalysis in der Studieneingangsphase}
\scAuthor{*}{Larissa}{Hahn}{1}
\scAuthor{}{Alexander}{Voigt}{2}
\scAuthor{}{Philipp}{Mertsch}{2}
\scAuthor{}{Pascal}{Klein}{1}
\scAffiliation{1}{Universität Göttingen, Deutschland}
\scAffiliation{2}{RWTH Aachen, Deutschland}
\scBeginText
%
Um Vektorfeldkonzepte wie Divergenz oder Rotation in physikalischen
Kontexten anzuwenden, ist ein konzeptionelles Verständnis
notwendig. Bisherige empirische Forschungsergebnisse bei Studierenden
zeigten hierbei Schwierigkeiten auf, die sich z. B. auf die visuelle
Interpretation von Richtungsableitungen zurückführen lassen.  Im
Einklang mit lerntheoretischen Erkenntnissen kann diesen
Schwierigkeiten mit dem Einsatz multipler Repräsentationen bei der
Vermittlung dieser Konzepte begegnet werden.  Zu diesem Zweck wurden
Lernaufgaben entwickelt, die einen visuellen Zugang zur Vektoranalysis
anhand von multiplen Repräsentationen ermöglichen und
Zeichenaktivitäten sowie ein interaktives
Vektorfeld-Visualisierungswerkzeug integrieren. Eine
Wirksamkeitsanalyse der Lernaufgaben durch Implementation in die
begleitenden Übungen einer Elektromagnetismus-Vorlesung an der
Universität Göttingen ergab höhere Lerneffekte der
multi-repräsentationalen Lernaufgaben im Vergleich zu traditionellen,
rechenbasierten Aufgaben ($N=81$). Eine Implemenation der Lernaufgaben
in die begleitenden Übungen einer Vorlesung zu mathematischen Methoden
der Physik im zweiten Studiensemester an der RWTH Aachen steht
bevor. Dieser Beitrag präsentiert zum einen die Lernaufgaben sowie
Ergebnisse der ersten Wirksamkeitsanalysen und stellt zum anderen die
Studie zur Implementation an der RWTH Aachen vor.
%
\scEndText
\scConference{Göttingen 2025}
\scPart{DD}
\scContributionType{Poster}
\scTopic{Hochschuldidaktik}
\scKeywords{Lernaufgaben; Vektoranalysis;  multiple Repräsentationen; Implementation}
\scEmail{larissa.hahn@uni-goettingen.de}
\scCountry{Germany}
\end{scAbstract}
\end{document}
