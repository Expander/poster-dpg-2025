\documentclass{scdpg}
\begin{document}
\scBookLanguage{de}
\begin{scAbstract}
% \scNoUseTeX
\scLanguage{de}
\scTitle{Multi-repräsentationale Lernaufgaben zur Vektoranalysis in der Studieneingangsphase}
\scAuthor{*}{Larissa}{Hahn}{1}
\scAuthor{}{Alexander}{Voigt}{2}
\scAuthor{}{Philipp}{Mertsch}{2}
\scAuthor{}{Pascal}{Klein}{1}
\scAffiliation{1}{Universität Göttingen, Deutschland}
\scAffiliation{2}{RWTH Aachen, Deutschland}
\scBeginText
Um Vektorfeldkonzepte wie Divergenz oder Rotation in physikalischen Kontexten anzuwenden, ist ein solides Verständnis ihrer Grundlagen erforderlich. Bisherige empirische Forschungsergebnisse bei Studierenden zeigten hierbei Schwierigkeiten auf, die sich z. B. auf die visuelle Interpretation von Richtungsableitungen zurückführen lassen. Im Einklang mit lerntheoretischen Erkenntnissen wird daher der Einsatz multipler Repräsentationen bei der Vermittlung dieser Konzepte empfohlen. Auf Basis der empirischen Vorarbeiten wurden Lernaufgaben entwickelt, die einen visuellen Zugang zur Vektoranalysis anhand von multiplen Repräsentationen (MR) ermöglichen und Zeichenaktivitäten sowie ein interaktives Vektorfeld-Visualisierungswerkzeug integrieren. Diese MR-Lernaufgaben wurden in die begleitenden Übungen einer Elektromagnetismus-Vorlesung an der Universität Göttingen implementiert ($N=81$). Die Wirksamkeitsanalyse ergab höhere Lerneffekte der MR-Lernaufgaben im Vergleich zu traditionellen, rechenbasierten Aufgaben. Eine Implemenation der Lernaufgaben in die begleitenden Übungen einer Vorlesung zu mathematischen Methoden der Physik im zweiten Studiensemester an der RWTH Aachen steht bevor. Dieser Beitrag präsentiert zum einen die Lernaufgaben sowie Ergebnisse der ersten Wirksamkeitsanalysen und stellt zum anderen die Studie zur Implementation an der RWTH Aachen vor.
\scEndText
\scConference{Göttingen 2025}
\scPart{DD}
\scContributionType{Poster}
\scTopic{Hochschuldidaktik}
\scKeywords{Lernaufgaben; Vektoranalysis;  multiple Repräsentationen; Implementation}
\scEmail{larissa.hahn@uni-goettingen.de}
\scCountry{Germany}
\end{scAbstract}
\end{document}
